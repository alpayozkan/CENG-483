\documentclass[12pt]{article}
\usepackage[utf8]{inputenc}
\usepackage[dvips]{graphicx}
\usepackage{epsfig}
\usepackage{fancybox}
\usepackage{verbatim}
\usepackage{array}
\usepackage{latexsym}
\usepackage{alltt}
\usepackage{amssymb}
\usepackage{amsmath}
\usepackage{hyperref}
\usepackage{listings}
\usepackage{color}
\usepackage{algorithm}
\usepackage{algpseudocode}
\usepackage[hmargin=3cm,vmargin=5.0cm]{geometry}
\usepackage{epstopdf}
\topmargin=-1.8cm
\addtolength{\textheight}{6.5cm}
\addtolength{\textwidth}{2.0cm}
\setlength{\oddsidemargin}{0.0cm}
\setlength{\evensidemargin}{0.0cm}
\newcommand{\HRule}{\rule{\linewidth}{1mm}}
\newcommand{\kutu}[2]{\framebox[#1mm]{\rule[-2mm]{0mm}{#2mm}}}
\newcommand{\gap}{ \\[1mm] }
\newcommand{\Q}{\raisebox{1.7pt}{$\scriptstyle\bigcirc$}}
\newcommand{\minus}{\scalebox{0.35}[1.0]{$-$}}



\lstset{
    %backgroundcolor=\color{lbcolor},
    tabsize=2,
    language=MATLAB,
    basicstyle=\footnotesize,
    numberstyle=\footnotesize,
    aboveskip={0.0\baselineskip},
    belowskip={0.0\baselineskip},
    columns=fixed,
    showstringspaces=false,
    breaklines=true,
    prebreak=\raisebox{0ex}[0ex][0ex]{\ensuremath{\hookleftarrow}},
    %frame=single,
    showtabs=false,
    showspaces=false,
    showstringspaces=false,
    identifierstyle=\ttfamily,
    keywordstyle=\color[rgb]{0,0,1},
    commentstyle=\color[rgb]{0.133,0.545,0.133},
    stringstyle=\color[rgb]{0.627,0.126,0.941},
}


\begin{document}

\noindent
\HRule %\\[3mm]
\small
\begin{center}
	\LARGE \textbf{CENG 483} \\[4mm]
	\Large Introduction to Computer Vision \\[4mm]
	\normalsize Fall 2021-2022 \\
	\Large Take Home Exam 1 \\
	\Large Instance Recognition with Color Histograms \\
    \Large Student ID: \\
\end{center}
\HRule

\begin{center}
\end{center}
\vspace{-10mm}
\noindent\\ \\ 
Please fill in the sections below only with the requested information. If you have additional things you want to mention, you can use the last section. For all of the configurations make sure that your 
quantization interval is divisible by 256 in order to obtain equal bins.
\section{3D Color Histogram}
In this section, give your results without dividing the images into grids. Your histogram must have at most 4096 bins. E.g. Assume that you choose 16 for quantization interval then you will have 16 bins for each channel and 4096 bins for your 3D color histogram.

\begin{itemize}
\item Pick 4 different quantization intervals and give your top-1 accuracy results for each of them on every query dataset.
\item Explain the differences in results and possible causes of them if there are any.
\end{itemize}


\begin{tabular}{ |p{1.5cm}|p{1.5cm}||p{2cm}|p{2cm}|p{2cm}|p{2cm}|  }
    \hline
    \multicolumn{6}{|c|}{Results} \\
    \hline
    Interval & Bins & Query-1 & Query-2 & Query-3 & Support \\
    \hline
    16  & 16 & 1.0 & 1.0 & 0.11 & 1.0 \\
    \hline
    32  & 8 & 1.0 & 1.0 & 0.11 & 1.0 \\
    \hline
    64  & 4 & 1.0  & 1.0 & 0.12 & 1.0 \\
    \hline
    128 & 2 & 0.935 & 1.0 &  0.085 & 1.0 \\
    \hline
\end{tabular}

\vspace{1cm}


Query-1 dataset includes the same images as in Support dataset, but it's scaled up (zoomed in), 
birds are larger in the pictures.
Since scaling is not very extreme, histogram was sucessfull to capture the essential
pixel distribution for matching the pictures. Query-1 is perfectly detected for all interval sizes 
except for (128x2) which has larger interval size corresponding to lower spatiality 
in terms of pixel values. However, 0.935 of accuracy is still successfull in Query-1.

Query-2 dataset also includes the same images, but with rotations (90, 180, 270 degrees).
Since we are calculating histogram for the full image without any spatial partitions/grids
and it only counts for the frequencies of pixel values, histogram is invariant to rotation in this case.
Therefore, as expected Query-2 set perfectly matches to Support set.

Query-3 dataset is composed of same images as in Support dataset without rotation or scaling,
but different transformations are applied on these images altering the pixel values. 
For example, American Pipit turns into yellow from brown. Those transformations might 
be contrast and hue transformations which can change the pixel values. 
As it's clear from the Query-3 set, applying an entire histogram over images wouldn't be sufficient 
to capture its correspondance in Support set. Altough semantically the images correspond to the 
same  entity, modified pixel values deviate/fool our full histogram approach resulting in poor accuracies
for all interval sizes.


Additionally, I've included Support to Support matching for 
sanity check and comparison as a baseline. As it's clear, it perfectly matches with itself.


\section{Per Channel Color histogram}
In this section, give your results without dividing the images into grids.

\begin{itemize}
\item Pick 5 different quantization intervals and give your top-1 accuracy results for each of them on every query dataset.
\item Explain the differences in results and possible causes of them if there are any.
\end{itemize}

\begin{tabular}{ |p{1.5cm}|p{1.5cm}||p{2cm}|p{2cm}|p{2cm}|p{2cm}|  }
    \hline
    \multicolumn{6}{|c|}{Results} \\
    \hline
    Interval & Bins & Query-1 & Query-2 & Query-3 & Support \\
    \hline
    8 & 32 & 0.98 & 1.0 & 0.125 & 1.0 \\
    \hline
    16 & 16 & 0.98 & 1.0 & 0.12 & 1.0 \\
    \hline
    32 & 8 & 0.98 & 1.0 & 0.135 & 1.0 \\
    \hline
    64 & 4 & 0.935 & 1.0 & 0.14 & 1.0 \\
    \hline
    128 & 2 & 0.585 & 0.995 & 0.04 & 1.0 \\
    \hline
\end{tabular}

\vspace{1cm}


\newpage
\textbf{Before starting the next section, please pick up the best configuration for two properties above and continue with them.}

\section{Grid Based Feature Extraction - Query set 1}
Give your top-1 accuracy for all of the configurations below.

% temporary table to gather overall results
% remove after separate insertions to each subsection

\begin{tabular}{ |p{1.5cm}||p{2cm}|p{2cm}|p{2cm}|p{2cm}|  }
    \hline
    \multicolumn{5}{|c|}{Results Spatial Grid: 3d histogram (intv=64)} \\
    \hline
    Grid & Query-1 & Query-2 & Query-3 & Support \\
    \hline
    48 x 48 & 1.0 & 0.555 & 0.155 & 1.0 \\
    \hline
    24 x 24 & 1.0 & 0.36 & 0.22 & 1.0 \\
    \hline
    16 x 16 & 1.0 & 0.3 & 0.255 & 1.0 \\
    \hline
    12 x 12 & 1.0 & 0.31 & 0.3 & 1.0 \\
    \hline
\end{tabular}

\vspace*{1cm}

\begin{tabular}{ |p{1.5cm}||p{2cm}|p{2cm}|p{2cm}|p{2cm}|  }
    \hline
    \multicolumn{5}{|c|}{Results Spatial Grid: per-channel histogram (intv=32)} \\
    \hline
    Grid & Query-1 & Query-2 & Query-3 & Support \\
    \hline
    48 x 48 & 1.0 & 0.36 & 0.22 & 1.0 \\
    \hline
    24 x 24 & 1.0 & 0.205 & 0.26 & 1.0 \\
    \hline
    16 x 16 & 1.0 & 0.13 & 0.27 & 1.0 \\
    \hline
    12 x 12 & 0.995 & 0.1 & 0.265 & 1.0 \\
    \hline
\end{tabular}

\subsection{$48\times48$ spatial grid}
\begin{itemize}
\item 3d color histogram:
\item per-channel histogram:
\end{itemize}

\subsection{$24\times24$ spatial grid}
\begin{itemize}
\item 3d color histogram:
\item per-channel histogram:
\end{itemize}

\subsection{$16\times16$ spatial grid}
\begin{itemize}
\item 3d color histogram:
\item per-channel histogram:
\end{itemize}

\subsection{$12\times12$ spatial grid}
\begin{itemize}
\item 3d color histogram:
\item per-channel histogram:
\end{itemize}

\subsection{Questions}
\begin{itemize}
\item What do you think about the cause of the difference between the results?
\item Explain the advantages/disadvantages of using grids in both types of histograms if there are any.
\end{itemize}

\section{Grid Based Feature Extraction - Query set 2}
Give your top-1 accuracy for all of the configurations below.

\subsection{$48\times48$ spatial grid}
\begin{itemize}
\item 3d color histogram:
\item per-channel histogram:
\end{itemize}

\subsection{$24\times24$ spatial grid}
\begin{itemize}
\item 3d color histogram:
\item per-channel histogram:
\end{itemize}

\subsection{$16\times16$ spatial grid}
\begin{itemize}
\item 3d color histogram:
\item per-channel histogram:
\end{itemize}

\subsection{$12\times12$ spatial grid}
\begin{itemize}
\item 3d color histogram:
\item per-channel histogram:
\end{itemize}

\subsection{Questions}
\begin{itemize}
\item What do you think about the cause of the difference between the results?
\item Explain the advantages/disadvantages of using grids in both types of histograms if there are any.
\end{itemize}


\section{Grid Based Feature Extraction - Query set 3}
Give your top-1 accuracy for all of the configurations below.

\subsection{$48\times48$ spatial grid}
\begin{itemize}
\item 3d color histogram:
\item per-channel histogram:
\end{itemize}

\subsection{$24\times24$ spatial grid}
\begin{itemize}
\item 3d color histogram:
\item per-channel histogram:
\end{itemize}

\subsection{$16\times16$ spatial grid}
\begin{itemize}
\item 3d color histogram:
\item per-channel histogram:
\end{itemize}

\subsection{$12\times12$ spatial grid}
\begin{itemize}
\item 3d color histogram:
\item per-channel histogram:
\end{itemize}

\subsection{Questions}
\begin{itemize}
\item What do you think about the cause of the difference between the results?
\item Explain the advantages/disadvantages of using grids in both types of histograms if there are any.
\end{itemize}


\section{Additional Comments and References}

(if there any)





\end{document}

