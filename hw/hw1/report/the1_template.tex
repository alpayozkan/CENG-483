\documentclass[12pt]{article}
\usepackage[utf8]{inputenc}
\usepackage[dvips]{graphicx}
\usepackage{epsfig}
\usepackage{fancybox}
\usepackage{verbatim}
\usepackage{array}
\usepackage{latexsym}
\usepackage{alltt}
\usepackage{amssymb}
\usepackage{amsmath}
\usepackage{hyperref}
\usepackage{listings}
\usepackage{color}
\usepackage{algorithm}
\usepackage{algpseudocode}
\usepackage[hmargin=3cm,vmargin=5.0cm]{geometry}
\usepackage{epstopdf}
\topmargin=-1.8cm
\addtolength{\textheight}{6.5cm}
\addtolength{\textwidth}{2.0cm}
\setlength{\oddsidemargin}{0.0cm}
\setlength{\evensidemargin}{0.0cm}
\newcommand{\HRule}{\rule{\linewidth}{1mm}}
\newcommand{\kutu}[2]{\framebox[#1mm]{\rule[-2mm]{0mm}{#2mm}}}
\newcommand{\gap}{ \\[1mm] }
\newcommand{\Q}{\raisebox{1.7pt}{$\scriptstyle\bigcirc$}}
\newcommand{\minus}{\scalebox{0.35}[1.0]{$-$}}



\lstset{
    %backgroundcolor=\color{lbcolor},
    tabsize=2,
    language=MATLAB,
    basicstyle=\footnotesize,
    numberstyle=\footnotesize,
    aboveskip={0.0\baselineskip},
    belowskip={0.0\baselineskip},
    columns=fixed,
    showstringspaces=false,
    breaklines=true,
    prebreak=\raisebox{0ex}[0ex][0ex]{\ensuremath{\hookleftarrow}},
    %frame=single,
    showtabs=false,
    showspaces=false,
    showstringspaces=false,
    identifierstyle=\ttfamily,
    keywordstyle=\color[rgb]{0,0,1},
    commentstyle=\color[rgb]{0.133,0.545,0.133},
    stringstyle=\color[rgb]{0.627,0.126,0.941},
}


\begin{document}

\noindent
\HRule %\\[3mm]
\small
\begin{center}
	\LARGE \textbf{CENG 483} \\[4mm]
	\Large Introduction to Computer Vision \\[4mm]
	\normalsize Fall 2021-2022 \\
	\Large Take Home Exam 1 \\
	\Large Instance Recognition with Color Histograms \\
    \Large Student ID: \\
\end{center}
\HRule

\begin{center}
\end{center}
\vspace{-10mm}
\noindent\\ \\ 
Please fill in the sections below only with the requested information. If you have additional things you want to mention, you can use the last section. For all of the configurations make sure that your 
quantization interval is divisible by 256 in order to obtain equal bins.
\section{3D Color Histogram}
In this section, give your results without dividing the images into grids. Your histogram must have at most 4096 bins. E.g. Assume that you choose 16 for quantization interval then you will have 16 bins for each channel and 4096 bins for your 3D color histogram.

\begin{itemize}
\item Pick 4 different quantization intervals and give your top-1 accuracy results for each of them on every query dataset.
\item Explain the differences in results and possible causes of them if there are any.
\end{itemize}


\begin{tabular}{ |p{1.5cm}|p{1.5cm}||p{2cm}|p{2cm}|p{2cm}|p{2cm}|  }
    \hline
    \multicolumn{6}{|c|}{Results} \\
    \hline
    Interval & Bins & Query-1 & Query-2 & Query-3 & Support \\
    \hline
    16  & 16 & 1.0 & 1.0 & 0.11 & 1.0 \\
    \hline
    32  & 8 & 1.0 & 1.0 & 0.11 & 1.0 \\
    \hline
    64  & 4 & 1.0  & 1.0 & 0.12 & 1.0 \\
    \hline
    128 & 2 & 0.935 & 1.0 &  0.085 & 1.0 \\
    \hline
\end{tabular}

\vspace{1cm}


Query-1 dataset includes scaled versions of images in Support dataset, 
ie. birds are larger in the pictures.
Since scaling is not very extreme, histogram was sucessful to capture the essential
pixel distribution for matching the pictures. Query-1 is perfectly detected for all interval sizes 
except for (128x2) which has larger interval size corresponding to lower spatiality 
in terms of pixel values. However, 0.935 of accuracy can still be considered as successfull in Query-1.
For intv=128, there is actually a problem and it would be even much severe for the case of intv$>$128 or intv=256 for the extreme case.
It's because of the fact that, scaling up emphasizes bird pixels more, namely there are more bird pixels 
rather than background pixels as in the Support set. Increasing the interval size 
decreases the number of bins that pixels can be matched with, hence increasing the difference between
support and query-1 as a result of the different pixel density between bird and background.

Query-2 dataset also includes the same images, but with rotations (90, 180, 270 degrees).
Since we are calculating histogram for the full image without any spatial partitions/grids
and it only counts for the frequencies of pixel values, histogram is invariant to rotation in this case.
Therefore, as expected Query-2 set perfectly matches to Support set.
Note that, in query-2 increasing interval size doesn't decrease the performance because pixel density 
between bird and background is the same for both rotated and original sets.

Query-3 dataset is composed of same images as in Support dataset without rotation or scaling,
but different transformations are applied on these images altering the pixel values. 
For example, American Pipit turns into yellow from brown. Those transformations might 
be contrast and hue transformations which can alter the pixel values. 
As it's clear from the Query-3 set, applying an entire histogram over images wouldn't be sufficient 
to capture its correspondance in Support set. Altough semantically the images correspond to the 
same  entity, modified pixel values deviate/fool our full histogram approach resulting in poor accuracies
for all interval sizes.


Additionally, I've included Support to Support matching for 
sanity check and comparison as a baseline. As it's clear, it perfectly matches with itself.


\section{Per Channel Color histogram}
In this section, give your results without dividing the images into grids.

\begin{itemize}
\item Pick 5 different quantization intervals and give your top-1 accuracy results for each of them on every query dataset.
\item Explain the differences in results and possible causes of them if there are any.
\end{itemize}

\begin{tabular}{ |p{1.5cm}|p{1.5cm}||p{2cm}|p{2cm}|p{2cm}|p{2cm}|  }
    \hline
    \multicolumn{6}{|c|}{Results} \\
    \hline
    Interval & Bins & Query-1 & Query-2 & Query-3 & Support \\
    \hline
    8 & 32 & 0.98 & 1.0 & 0.125 & 1.0 \\
    \hline
    16 & 16 & 0.98 & 1.0 & 0.12 & 1.0 \\
    \hline
    32 & 8 & 0.98 & 1.0 & 0.135 & 1.0 \\
    \hline
    64 & 4 & 0.935 & 1.0 & 0.14 & 1.0 \\
    \hline
    128 & 2 & 0.585 & 0.995 & 0.04 & 1.0 \\
    \hline
\end{tabular}

\vspace{1cm}

Overall accuracies are worse than 3D histogram. This is expected because
3D histogram can represent more color combinations, preserving color value spatiality.
For example, for intv=16 3D histogram has 16x16x16=4096 bins for (R,G,B) to associate pixels
whereas per-channel histogram has 16 bins for each channel (R/G/B). In configurations with decreased bin size
we expect less accuracy because pixels will be associated/mapped with less choices
causing in a sense information loss. The extreme case would be only 1 bin per channel
which can be easily fooled, because every histogram will be the same 
(each pixel fits into 0-255 interval for each channel) regardless of the input 
that generated that histogram.

Overall accuracies decrease for all queries as interval size increases 
as a result decreasing spatial or local color characteristics.

Query-1 accuracy decreases dramatically for intv=128 with similar reasoning 
as in 3D histograms background and bird pixel density. However, it decreases
much severely compared to 3D histogram which is due to color histogram's weakness
for less color spatiality.

Query-2 results are all almost correct except for the intv=128 which was 
1.0 in 3D histogram. Once again, per-channel histograms accuracy starts to decrease earlier
than 3D histogram's. If we had 256x1 configuration, accuracy loss would be much more dramatic.
Query-2 accuracies are higher than Query-1's because of the fact that 
Query-2 includes the same pixels as in Support set, whereas Query-1 has
more bird pixels which doesn't help the accuracy because our base is Support set
with which we measure its KL divergence or distribution similarity.

Query-3 results are still very low, mainly due to the same reasoning 
as in 3D histogram for this case. In short, pixel values are significantly different.
For example, some images are darkened and some brightened: contrast, and hue transformations.
These transformations assign pixels different bins compared to support set's.
Compared to 3D histogram both methods more or less fail around similar orders of magnitude.
However, for this case spatial partitions on images can make a difference which I suspect
will increase our representation power in the sense that parts of different images
will have more unlikely similarities when cumulated in a histogram.


\textbf{Before starting the next section, please pick up the best configuration for two properties above and continue with them.}

\section{Grid Based Feature Extraction - Query set 1}
Give your top-1 accuracy for all of the configurations below.

% temporary table to gather overall results
% remove after separate insertions to each subsection

\begin{tabular}{ |p{1.5cm}||p{2cm}|p{2cm}|p{2cm}|p{2cm}|  }
    \hline
    \multicolumn{5}{|c|}{Results Spatial Grid: 3d histogram (intv=64)} \\
    \hline
    Grid & Query-1 & Query-2 & Query-3 & Support \\
    \hline
    48 x 48 & 1.0 & 0.555 & 0.155 & 1.0 \\
    \hline
    24 x 24 & 1.0 & 0.36 & 0.22 & 1.0 \\
    \hline
    16 x 16 & 1.0 & 0.3 & 0.255 & 1.0 \\
    \hline
    12 x 12 & 1.0 & 0.31 & 0.3 & 1.0 \\
    \hline
\end{tabular}

\vspace*{1cm}

\begin{tabular}{ |p{1.5cm}||p{2cm}|p{2cm}|p{2cm}|p{2cm}|  }
    \hline
    \multicolumn{5}{|c|}{Results Spatial Grid: per-channel histogram (intv=32)} \\
    \hline
    Grid & Query-1 & Query-2 & Query-3 & Support \\
    \hline
    48 x 48 & 1.0 & 0.36 & 0.22 & 1.0 \\
    \hline
    24 x 24 & 1.0 & 0.205 & 0.26 & 1.0 \\
    \hline
    16 x 16 & 1.0 & 0.13 & 0.27 & 1.0 \\
    \hline
    12 x 12 & 0.995 & 0.1 & 0.265 & 1.0 \\
    \hline
\end{tabular}

\subsection{$48\times48$ spatial grid}
\begin{itemize}
\item 3d color histogram:
\item per-channel histogram:
\end{itemize}

\subsection{$24\times24$ spatial grid}
\begin{itemize}
\item 3d color histogram:
\item per-channel histogram:
\end{itemize}

\subsection{$16\times16$ spatial grid}
\begin{itemize}
\item 3d color histogram:
\item per-channel histogram:
\end{itemize}

\subsection{$12\times12$ spatial grid}
\begin{itemize}
\item 3d color histogram:
\item per-channel histogram:
\end{itemize}

\subsection{Questions}
\begin{itemize}
\item What do you think about the cause of the difference between the results?
\item Explain the advantages/disadvantages of using grids in both types of histograms if there are any.
\end{itemize}

\section{Grid Based Feature Extraction - Query set 2}
Give your top-1 accuracy for all of the configurations below.

\subsection{$48\times48$ spatial grid}
\begin{itemize}
\item 3d color histogram:
\item per-channel histogram:
\end{itemize}

\subsection{$24\times24$ spatial grid}
\begin{itemize}
\item 3d color histogram:
\item per-channel histogram:
\end{itemize}

\subsection{$16\times16$ spatial grid}
\begin{itemize}
\item 3d color histogram:
\item per-channel histogram:
\end{itemize}

\subsection{$12\times12$ spatial grid}
\begin{itemize}
\item 3d color histogram:
\item per-channel histogram:
\end{itemize}

\subsection{Questions}
\begin{itemize}
\item What do you think about the cause of the difference between the results?
\item Explain the advantages/disadvantages of using grids in both types of histograms if there are any.
\end{itemize}


\section{Grid Based Feature Extraction - Query set 3}
Give your top-1 accuracy for all of the configurations below.

\subsection{$48\times48$ spatial grid}
\begin{itemize}
\item 3d color histogram:
\item per-channel histogram:
\end{itemize}

\subsection{$24\times24$ spatial grid}
\begin{itemize}
\item 3d color histogram:
\item per-channel histogram:
\end{itemize}

\subsection{$16\times16$ spatial grid}
\begin{itemize}
\item 3d color histogram:
\item per-channel histogram:
\end{itemize}

\subsection{$12\times12$ spatial grid}
\begin{itemize}
\item 3d color histogram:
\item per-channel histogram:
\end{itemize}

\subsection{Questions}
\begin{itemize}
\item What do you think about the cause of the difference between the results?
\item Explain the advantages/disadvantages of using grids in both types of histograms if there are any.
\end{itemize}


\section{Additional Comments and References}

Additional Grid Based Feature Extraction for different intervals besides the best pick. \\



\textbf{3D - Histogram} \\ 

\begin{tabular}{ |p{1.5cm}||p{2cm}|p{2cm}|p{2cm}|p{2cm}|  }
    \hline
    \multicolumn{5}{|c|}{Results 3D Grid histogram (intv=16)} \\
    \hline
    Grid & Query-1 & Query-2 & Query-3 & Support \\
    \hline
    48 x 48 & 1.0 & 0.71 & 0.135 & 1.0 \\
    \hline
    24 x 24 & 1.0 & 0.565 & 0.16 & 1.0 \\
    \hline
    16 x 16 & 1.0 & 0.545 & 0.215 & 1.0 \\
    \hline
    12 x 12 & 1.0 & 0.555 & 0.255 & 1.0 \\
    \hline
\end{tabular}

\begin{tabular}{ |p{1.5cm}||p{2cm}|p{2cm}|p{2cm}|p{2cm}|  }
    \hline
    \multicolumn{5}{|c|}{Results 3D Grid histogram (intv=32)} \\
    \hline
    Grid & Query-1 & Query-2 & Query-3 & Support \\
    \hline
    48 x 48 & 1.0 & 0.655 & 0.15 & 1.0 \\
    \hline
    24 x 24 & 1.0 & 0.425 & 0.155 & 1.0 \\
    \hline
    16 x 16 & 1.0 & 0.4 & 0.22 & 1.0 \\
    \hline
    12 x 12 & 1.0 & 0.39 & 0.245 & 1.0 \\
    \hline
\end{tabular}

\begin{tabular}{ |p{1.5cm}||p{2cm}|p{2cm}|p{2cm}|p{2cm}|  }
    \hline
    \multicolumn{5}{|c|}{Results 3D Grid histogram (intv=64)} \\
    \hline
    Grid & Query-1 & Query-2 & Query-3 & Support \\
    \hline
    48 x 48 & 1.0 & 0.555 & 0.155 & 1.0 \\
    \hline
    24 x 24 & 1.0 & 0.36 & 0.22 & 1.0 \\
    \hline
    16 x 16 & 1.0 & 0.3 & 0.255 & 1.0 \\
    \hline
    12 x 12 & 1.0 & 0.31 & 0.3 & 1.0 \\
    \hline
\end{tabular}

\begin{tabular}{ |p{1.5cm}||p{2cm}|p{2cm}|p{2cm}|p{2cm}|  }
    \hline
    \multicolumn{5}{|c|}{Results 3D Grid histogram (intv=128)} \\
    \hline
    Grid & Query-1 & Query-2 & Query-3 & Support \\
    \hline
    48 x 48 & 0.995 & 0.28 & 0.15 & 1.0 \\
    \hline
    24 x 24 & 1.0 & 0.195 & 0.24 & 1.0 \\
    \hline
    16 x 16 & 0.995 & 0.14 & 0.285 & 1.0 \\
    \hline
    12 x 12 & 1.0 & 0.145 & 0.34 & 1.0 \\
    \hline
\end{tabular}

\pagebreak

\textbf{Per-Channel Histogram} \\ 

\begin{tabular}{ |p{1.5cm}||p{2cm}|p{2cm}|p{2cm}|p{2cm}|  }
    \hline
    \multicolumn{5}{|c|}{Results Per-Channel Grid histogram (intv=8)} \\
    \hline
    Grid & Query-1 & Query-2 & Query-3 & Support \\
    \hline
    48 x 48 & 1.0 & 0.355 & 0.215 & 1.0 \\
    \hline
    24 x 24 & 1.0 & 0.16 & 0.22 & 1.0 \\
    \hline
    16 x 16 & 0.995 & 0.1 & 0.24 & 1.0 \\
    \hline
    12 x 12 & 0.995 & 0.1 & 0.24 & 1.0 \\
    \hline
\end{tabular}

\begin{tabular}{ |p{1.5cm}||p{2cm}|p{2cm}|p{2cm}|p{2cm}|  }
    \hline
    \multicolumn{5}{|c|}{Results Per-Channel Grid histogram (intv=16)} \\
    \hline
    Grid & Query-1 & Query-2 & Query-3 & Support \\
    \hline
    48 x 48 & 1.0 & 0.365 & 0.21 & 1.0 \\
    \hline
    24 x 24 & 1.0 & 0.17 & 0.225 & 1.0 \\
    \hline
    16 x 16 & 1.0 & 0.115 & 0.25 & 1.0 \\
    \hline
    12 x 12 & 0.995 & 0.11 & 0.245 & 1.0 \\
    \hline
\end{tabular}

\begin{tabular}{ |p{1.5cm}||p{2cm}|p{2cm}|p{2cm}|p{2cm}|  }
    \hline
    \multicolumn{5}{|c|}{Results Per-Channel Grid histogram (intv=32)} \\
    \hline
    Grid & Query-1 & Query-2 & Query-3 & Support \\
    \hline
    48 x 48 & 1.0 & 0.36 & 0.22 & 1.0 \\
    \hline
    24 x 24 & 1.0 & 0.205 & 0.26 & 1.0 \\
    \hline
    16 x 16 & 1.0 & 0.13 & 0.27 & 1.0 \\
    \hline
    12 x 12 & 0.995 & 0.1 & 0.265 & 1.0 \\
    \hline
\end{tabular}

\begin{tabular}{ |p{1.5cm}||p{2cm}|p{2cm}|p{2cm}|p{2cm}|  }
    \hline
    \multicolumn{5}{|c|}{Results Per-Channel Grid histogram (intv=64)} \\
    \hline
    Grid & Query-1 & Query-2 & Query-3 & Support \\
    \hline
    48 x 48 & 1.0 & 0.33 & 0.235 & 1.0 \\
    \hline
    24 x 24 & 1.0 & 0.2 & 0.315 & 1.0 \\
    \hline
    16 x 16 & 0.995 & 0.125 & 0.315 & 1.0 \\
    \hline
    12 x 12 & 0.995 & 0.1 & 0.305 & 1.0 \\
    \hline
\end{tabular}

\begin{tabular}{ |p{1.5cm}||p{2cm}|p{2cm}|p{2cm}|p{2cm}|  }
    \hline
    \multicolumn{5}{|c|}{Results Per-Channel Grid histogram (intv=128)} \\
    \hline
    Grid & Query-1 & Query-2 & Query-3 & Support \\
    \hline
    48 x 48 & 0.97 & 0.175 & 0.21 & 1.0 \\
    \hline
    24 x 24 & 1.0 & 0.125 & 0.355 & 1.0 \\
    \hline
    16 x 16 & 0.985 & 0.1 & 0.395 & 1.0 \\
    \hline
    12 x 12 & 0.98 & 0.065 & 0.405 & 1.0 \\
    \hline
\end{tabular}

\end{document}

